%% LyX 1.3 created this file.  For more info, see http://www.lyx.org/.
%% Do not edit unless you really know what you are doing.
\documentclass[english]{article}
\usepackage[T1]{fontenc}
\usepackage[latin1]{inputenc}

\makeatletter
\usepackage{babel}
\makeatother
\begin{document}

\section{CCopasiObject}

Many classes in Copasi are derived of the CCopasiObject class. They
all inherit some common properties:

\begin{itemize}
\item All Objects have a name and a type
\item The objects can be organized in a hierarchical tree
\item There is a mechanism to reference objects (Common Names)
\item Objects can have values that can be calculated and queried with a
unified mechanism
\item All output in Copasi is handled with mechanisms of the CCopasiObject
class
\end{itemize}
The following is a more detailed description of those properties.


\subsection{Name and Type}

Every CCopasiObject instance has a name and a type. Both are STL strings.

The name can be set with setObjectName() and can be accessed with
getObjectName(). The type is set in the constructor of the derived
classes (so there is no need for a setObjectType() method) and can
be accessed with the getObjectType() method.


\subsection{Hierarchical tree}

CCopasiObjects can be organized in a hierarchical tree. The parent
of an object is retrieved with getObjectParent(). In most cases when
an object is created the parent is passed to the constructor. Usually
it should not be necessary to call setObjectParent() directly. 

getObjectAncestor(std::string type) looks for an object with a given
type among the ancestors (parent, parent of parent, etc.) of a given
object.

An object can only have children if its class is derived from CCopasiContainer
(which is itself derived from CCopasiObject). For information on how
to access children of objects see the documentation of CCopasiContainer.

In the current Copasi implementation there is one single hierarchical
tree of CCopasiObjects. Most CCopasiObjects (except some temporary
ones) are part of this tree. The root of this tree is the static pointer
CCopasiContainer::Root. There is also a define which allows to access
the root container as RootContainer.


\subsection{Object references with common names (CNs)}

All CCopasiObjects in a hierarchical tree can be referenced by a so
called common name (CN). A CN is unique within the tree. The CN reflects
the position of the given object in the tree. It is a STL string and
is composed of the types and names of all the direct ancestors of
the object. {[}insert a description of how CNs actually work{]}

The CN is determined uniquely from the type and name of the object
and from its position in a tree. It does not depend on where in the
memory the object is located (like pointers would). The object doesn't
even need to exist to have a valid CN. Therefore CNs can be used (and
are indeed used) to reference CCopasiObjects in copasi files.

The structure of CNs also allows to identify corresponding objects
in different subtrees. 

The CN of an object is retrieved with the getCN() method. It returns
a CCopasiObjectName object. CCopasiObjectName is essentially an STL
string with some additional methods that help parsing the CN. 

To locate the object for a given CN use the static CCopasiContainer::ObjectFromName(cn)
method.

There is also a mechanism to handle renaming of objects (since the
CN is constructed using the object name it is usually invalidated
if an object is renamed): If a variable that holds a CN is declared
as a CRegisteredObjectName it will be automatically added to a global
static list of registered CNs. If an object that is refered to by
one of the registered CNs is renamed, the CN will be changed accordingly.


\subsection{Values of objects}

Some CCopasiObjects have a value. The current implementation allows
boolean, floating point, integer, and string values. The CCopasiObject
class provides a uniform interface to calculate and access those values.
Objects that have a floating point or integer value can be used in
a plot.


\subsubsection{Accessing the value}

The isValueBool(), isValueDbl(), isValueInt(), isValueString() methods
can be used to check if a given object has a value (it can only have
one value of one type). The getValuePointer() method returns a pointer
to this value. Since it returns a void{*} pointer it needs to be cast
to the appropriate type: bool, C\_FLOAT64, C\_INT32, std::string.

The value can be set using the setObjectValue(...) method (which is
declared for C\_FLOAT64, C\_INT32 and bool arguments).


\subsubsection{Update method}

Some objects need to do some extra actions when their value is set
(e.g. a CMetab needs to adjust the initial particle number if the
initial concentration is changed). CCopasiObject also provides an
interface for that. Every object can have an update method which takes
one argument (of the type corresponding to the value type of the object).
This method can be set using the setUpdateMethod({*}object, {*}member\_function).
It needs to be a member method of an object. Both the pointer to the
object and the member pointer of the method needs to be given to the
setUpdateMethod() method. Note that the update method can be a method
of a different object than the one that needs to be updated. In many
cases it will be the parent object. The update method should be set
in the constructor.


\subsubsection{Refresh method}

Some objects need to do some calculations to ensure their value is
uptodate. This is not done automatically by accessing the value (since
accessing the value usually happens through the pointer returned by
getValuePointer()). So the functor returned by getRefresh() needs
to be called explicitly before accessing the value of an object.

There may be some rules that specify that calling refresh is not necessary
in some circumstances. E.g. when calling output routines during a
calculation all variables of the model should already have an uptodate
value. But these rules have to be documented elsewhere.


\subsubsection{Dependencies}

The values of some objects depend on values of other objects. These
dependencies can be queried using the getDirectDependencies(), getAllDependencies(),
and hasCircularDependencies() methods. CCopasiObject also has a comparison
operator (operator<()) which can be used to sort a set of objects
according to their dependencies (so that the refresh methods can then
be called in the right order). I'm not sure this is implemented right
now (April 2006). 

The way this mechanism is used for output is that at the start of
a calculation task a list of all objects that need to be output is
generated. This list is (should be?) sorted according to the dependencies,
than a sorted list of refresh functors of this objects is stored.
At each output step the list of refresh functors is called in the
right order, after that the output can be done using the pointers
to the values (for plots) or the print() methods (for reports). By
doing this it can be avoided to call a refresh functor several times
if an object appears in the output several times. 


\subsection{Output in reports}

Each CCopasiObject has a print(std::ostream{*}) method which is used
for output in reports. Simple object that have a value just print
that value to the stream, more complex objects (like numerical methods)
generate a formatted output of their current settings or current state.

The refresh method (including the dependencies) should be called before
calling the print() method.


\subsection{CCopasiObjectReference}


\subsection{Hints for implementing derived classes}


\subsubsection{Constructors}

The standard constructor of a CCopasiObject has four arguments: name,
parent, type, and flags. The flags argument is an unsigned 32 bit
integer that contains some flags that specify certain properties of
objects (e.g. the type of its value, whether it is a container, ...).
The meaning of the different bits can be found in the CCopasiObject.h
header.

Constructors of derived classes should at least have the first two
arguments: name and parent. The other arguments are only necessary
if you expect that the class will be used as a base class for further
derived classes which may want to change the flags or type.


\section{CModelEntity}
\end{document}
