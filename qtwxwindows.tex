\documentclass[12pt,a4paper]{article}

\pagestyle{empty}

\begin{document}

\title{Comparison of WxWindows and Qt}
\date{}
\maketitle

\thispagestyle{empty}

\section*{Qt}
\begin{description}
\item[Portability:]
X11/Posix, Windows, OS X soon. \\
MacOS 9 not supported, probably will not be.
Windows 3.1 not supported.
\item[Language:] Primarily C++, but C, Perl and Python bindings are
  available.
\item[Completeness:]
Full featured and mature.
\item[Look and Feel:] Professional looking. Implements widgets on each
  platform to match native platform widget look.
\item[Ease of Use:] Very easy, integrates well with C++. Event
  handling is non-standard, requiring meta-object compiler, but this
  is easily learned, and probably less work overall than other
  event-handling mechanisms.
\item[Documentation:] Very well documented.
\item[Price:]Expensive (\$2000 to \$3000 per developer). Support and
  updates after first year must be bought (\$600-\$700 per developer).
\item[Licence:] Proprietary (no licence required for free software on
  Unix). GUI part of Windows software cannot be released later as open
  source.
\item[Bugginess:]
Few bugs.
\item[Updates:]Frequent, in first year.
\item[Support:]Commercially supported.
\item[Application Size:]Reimplements widgets on each platform (no use
  of native widgets), so size is bigger, but not huge.
\item[Speed:] Fast (Widgets implemented directly, no use of e.g. Xt
  library).

\end{description}

\section*{wxWindows}

\begin{description}
\item[Portability:]X11 (Motif, GTK), Win32/16. MacOS port underway but
  further behind.
\item[Language:] Primarily C++, but Perl, Python, XLisp, Scheme and
  CLIPS bindings are available.
\item[Completeness:] Less mature, some advanced features still lacking.
\item[Look and Feel:]Uses native widget kit on each platform. Possibly
  less professional looking.
\item[Ease of Use:] Similar to MFC, easy to learn and use.
\item[Documentation:] Well documented.
\item[Price:]Free.
\item[Licence:]LGPL, with extra provision that static binaries may be
  distributed.
\item[Bugginess:] Has some bugs.
\item[Updates:] At the whim of developers, but being open source,
  generally frequent.
\item[Support:]WWW, newsgroups.
\item[Application Size:] Smaller, since the native toolkit is used.
\item[Speed:] Slightly slower.

\end{description}

\section*{Summary}

There are a number of cross-platform widget kits available, among them
Qt, wxWindows, FLTK, FOX and V. Of these, only the first two provide
Mac support, and are reasonably full featured and mature.

wxWindows would be a good choice for cross-platform free software,
where it has an obvious price advantage, or if future support for
MacOS 9 were important (although at present Mac support is incomplete).
At the moment, however, wxWindows is less polished than Qt, in terms
of look and feel, feature completeness and bugginess. I recommend that
we use Qt as the widget kit for the COPASI front-end. The price is not
inconsiderable, but I think these are outweighed by Qt's maturity,
full feature set, professional look and ease of use.


\end{document}