\documentclass[a4,12pt]{article}
\usepackage[latin1]{inputenc}
\usepackage[ngerman]{babel}
\usepackage[usenames]{color}
\usepackage{listings}
\usepackage{hyperref}

\begin{document}
\lstset{ %
language=Java,                   % choose the language of the code
basicstyle=\footnotesize,       % the size of the fonts that are used for the code
keywordstyle=\color{black}\bfseries, % print keywords bold  
commentstyle=\color{blue},      % blue comments 
stringstyle=\ttfamily,          % typewriter type for strings 
showstringspaces=false,         % underline spaces within strings
numbers=left,                   % where to put the line-numbers
numberstyle=\tiny,              % the size of the fonts that are used for the line-numbers
stepnumber=1,                   % the step between two line-numbers. If it's 1 each line will be numbered
numbersep=5pt,                  % how far the line-numbers are from the code
backgroundcolor=\color{white},  % choose the background color. You must add \usepackage{color}
showspaces=false,               % show spaces within strings adding particular underscores
showtabs=false,                 % show tabs within strings adding particular underscores
escapeinside={\%*}{*)}          % if you want to add a comment within your code
}

\title{Documentation for the Java bindings for COPASI}
\author{Ralph Gauges}
\date{\today}
\maketitle
\parindent=0cm
\section{Working with the model}
\subsection{COPASIs Model Concept}
In COPASI, the top level class in the backend is CCopasiDataModel. This class contains the actual model as well as the tasks, the function database and all the output definitions (plots and reports). Right now, this class defines one global instance of itself and many methods in COPASIs backend rely on and work with this instance. Creating a second instance  of CCopasiDataModel therefore does not make much sense and will most likely lead to crashes sooner or later. If you work with the COPASI bindings, you are restricted to this one global instance which means that working with more than one model at any given time on one machine is not possible right now.
CCopasiDataModel provides the static method \textit{getGlobal()} to get the global instance of CCopasiDataModel.

\begin{lstlisting}
CCopasiDataModel globalModel=CCopasiDataModel.getGlobal();
\end{lstlisting}

\textit{CCopasiDataModel} 

\subsection{Model Elements}
\subsection{The Function Database}
\section{Working with Tasks}
\subsection{The Task-Problem-Method Concept in COPASI}
\subsection{Running Time Course Simulations}
\subsection{Running Optimizations}


\end{document}
